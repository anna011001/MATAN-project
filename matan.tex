\documentclass{article}

\usepackage[english]{babel}
\usepackage[letterpaper,top=2cm,bottom=2cm,left=3cm,right=3cm,marginparwidth=1.75cm]{geometry}
\usepackage{amsmath}
\usepackage{graphicx}
\usepackage{hyperref}
\usepackage{minted} % for including code with syntax highlighting

\title{PRONALAZAK MINIMUMA POMOĆU GRADIJENTA}
\author{Anna Bissinger, Dorijan Rijavec}
\date{}

\begin{document}
\maketitle

\begin{abstract}
U ovom radu analiziramo minimum funkcije koristeći gradijentni pristup i Hessianovu determinantu. Takođe prikazujemo graf razina funkcije koristeći Python.
\end{abstract}

\section{Zadatak 1}

Koristeći se testom drugih parcijalnih derivacija nađite ekstreme funkcije
f i odredite njihov karakter (minimum, maksimum, sedlasta točka). 

\subsection*{1. Parcijalne derivacije prvog reda}

\[
f_x = \frac{\partial f}{\partial x} = \frac{\partial}{\partial x} (4x^2 + y^2 + 4) = 8x
\]

\[
f_y = \frac{\partial f}{\partial y} = \frac{\partial}{\partial y} (4x^2 + y^2 + 4) = 2y
\]
\\
Gradijent funkcije \( f(x, y) \) je \( \nabla f = (8x, 2y) \).

\subsection*{2. Stacionarne točke}

Izjednačimo sve parcijalne derivacije prvog reda sa nulom i pronađemo nepoznanice.

\[
f_x = 0 \implies 8x = 0 \implies x = 0
\]
\[
f_y = 0 \implies 2y = 0 \implies y = 0
\]

Dakle, stacionarna točka je \((0, 0)\).

\subsection*{3. Parcijalne derivacije drugog reda}

Parcijalne derivacije drugog reda potrebne su nam za karakter.

\[
f_{xx} = \frac{\partial^2 f}{\partial x^2} = \frac{\partial}{\partial x} (8x) = 8 \cdot 1 = 8
\]
\[
f_{yy} = \frac{\partial^2 f}{\partial y^2} = \frac{\partial}{\partial y} (2y) = 2 \cdot 1 = 2
\]
\[
f_{xy} = \frac{\partial^2 f}{\partial x \partial y} = \frac{\partial}{\partial x} (2y) = 0
\]
\[
f_{yx} = \frac{\partial^2 f}{\partial y \partial x} = \frac{\partial}{\partial y} (8x) = 0
\]

\subsection*{4. Karakter dobivamo pomoću Hessianove determinante}

\[
\mathbf{D} = \begin{bmatrix}
f_{xx} & f_{xy} \\
f_{yx} & f_{yy}
\end{bmatrix} = \begin{bmatrix}
8 & 0 \\
0 & 2
\end{bmatrix} = 16 - 0^2 = 16
\]

Množili smo glavnu dijagonalu sa sporednom i dobili da je D = 16.\\

Budući da je \(\det(\mathbf{D}) = 16 > 0\) i \(f_{xx} = 8 > 0\), stacionarna točka \((0, 0)\) je \textbf{lokalni minimum}.

\section{Zadatak 2}

Grafički prikažite graf razina funkcije \( f(x, y) = 4x^2 + y^2 + 4 \).

\subsection*{Python kod za graf razina funkcije}

\begin{minted}[frame=lines, fontsize=\footnotesize, breaklines]{python}
import numpy as np
import matplotlib.pyplot as plt

# Definiramo funkciju
def f(x, y):
    return 4*x**2 + y**2 + 4

# Kreiranje mreže točaka u xy ravnini
x = np.linspace(-3, 3, 400)
y = np.linspace(-3, 3, 400)
X, Y = np.meshgrid(x, y)
Z = f(X, Y)

# Crtanje kontura
plt.figure(figsize=(10, 8)) 
contour = plt.contour(X, Y, Z, levels=10, cmap='viridis')
plt.clabel(contour, inline=1, fontsize=10)
plt.title('Konture funkcije $f(x, y) = 4x^2 + y^2 + 4$')
plt.xlabel('x')
plt.ylabel('y')
plt.grid(True)
plt.colorbar(contour)
plt.savefig('konture.png') 
plt.show()
\end{minted}

\begin{figure}[h]
    \centering
    \includegraphics[width=1.1\textwidth]{konture.png}
    \caption{Konture funkcije $f(x, y) = 4x^2 + y^2 + 4$}
    \label{fig:contour_plot}
\end{figure}

\end{document}